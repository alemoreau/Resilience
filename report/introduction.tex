\section{Introduction}

	The current trend to improve supercomputers' performance is to increase the component count while reducing their size. Future exascale systems will gather several millions of tiny CPU cores, which will drive the error rate up to many times per day. This issue is already taking place in some present petascale systems, that experience up to 3 failures per day (according to The Computer Failure Data Repository\cite{CFDR}, LANL). In case of a failure, an application without any fault tolerance mechanism may either stop or silently terminate with an incorrect output, which can have serious consequences in some applications. For this reason, we study the impact of faults and explore strategies to enhance the fault tolerance of particular applications in a specific context described below. 
  
  	Errors can be classified into two types, hard errors and soft errors. Hard errors are permanent and unrecoverable errors, for instance a system crash or a broken component, whereas soft errors are silent data corruptions that usually do not affect the system reliability, but may lead to incorrect results. In the following, we focus on the study of soft error tolerance.
  
	Several schemes have been designed to enhance HPC system resiliency to soft errors. On one hand, hardware mechanisms usually relying on the addition of redundancy in the circuits are being used, but may be too costly in terms of hardware and energy to be systematically applicable in the future. On the other hand, software schemes are often more flexible and provide fault tolerance at lower cost.
    
    Software schemes may be system-wide or at the application level. The standard system-wide method is Checkpoint/Restart: the system state is periodically saved into a safe memory storage, and a rollback is performed from a previous correct state whenever an incorrect state is detected. However this approach becomes expensive as the system size increases,  and may not be applicable in the long term as the mean time between failures (MTBF) gets closer and closer to the mean time to repair (MTTR). Moreover, it still requires a reliable fault detection mechanism. At the application level, this method is also being used and usually benefits from much smaller sized states, enabling more frequent checkpoints and shorter recovery times.
    
   Other approaches at the application level include Algorithm Base Fault Tolerance (ABFT) such as check-sum based techniques, introduced by Kuang-Hua Huang and J. A. Abraham in \cite{checksum} and usually provide fault tolerance at lower cost and hardware overhead than standard application level schemes. More recently, studies on the algorithms numerical properties as well as floating point analysis have been used to provide evidences and quantify the impact of any potential perturbation occurring during the algorithm execution. More information about the current state of the art of the resilience in HPC can be found in \cite{Cappello:2014:TER:2983586.2983587}.
   
   One of the most important and time-consuming kernel in numerical schemes is the solution of large sparse linear systems. One popular algorithm for solving such systems it the generalized minimum residual (GMRES) algorithm \cite{gmres}. 
   In this report, we analyze the impact of soft-errors on the convergence of GMRES and deduce a new, robust fault-detection mechanism.
   
   The rest of the report is organized as follows. In Section 2, the GMRES algorithm is described, in particular the variants we are interested in (full-GMRES, with and without preconditioner). In Section 3, the method, definitions and assumptions used throughout the numerical experiments are detailed. In Section 4, the impact of faults in GMRES is empirically studied. In Section 5, an analysis of the error introduced by faults is proposed, to quantify their impact and predict their influence on the convergence. In Section 6, an oracle-based detection scheme is described and evaluate. In Section 7, a more practical implementation of the detection scheme, based on some approximations, is proposed and evaluated. Finally a discussion on the results and concluding remarks are given.

