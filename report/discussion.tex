\section{Conclusion}
In this study, we observed how transient faults can impact GMRES executions. Since some faults may induce errors large enough to disrupt the convergence, we proposed an experimental and an analytical study to get a better understanding of this phenomenon, by adapting the theory of inexact and relaxed GMRES. With this knowledge, a theoretical criterion able to accurately detect such faults was derived. However, to be of practical use, the detection scheme had to be adapted, using some approximation that enabled its implementation. It was then evaluated and performed well for the inputs used. However it may present some limitations. First, the quality of the error approximation by check-sum might deteriorate as the system size increases, so further experiments involving larger matrices should be performed. Then, the threshold approximation depends on a empirical statement that do not benefit from much theoretical attention, hence it might not be accurate for all inputs. Also, its implementation depends on some assumptions, such as the necessity for the matrix A to be able to be encoded while it often comes only in the form of an operator.
Finally, future work could include the detection in both the SpMV operation and the preconditioner application for the preconditioned-GMRES case, an adaptation for variants of the GMRES algorithm such as pipelined-GMREs \cite{pipelined_gmres}, and a parallel implementation.
